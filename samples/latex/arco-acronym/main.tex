\documentclass[a4paper, 11pt]{article}

\usepackage[T1]{fontenc}
\usepackage[utf8]{inputenc}
\usepackage[spanish]{babel}
\usepackage{times}

\usepackage{graphicx}
\graphicspath{{figures/}}

\usepackage{arco-acronym}

\title{}

\begin{document}
\maketitle


\section{Lista de acrónimos}

{\small
\begin{acronym}[XXXXXXXX]
  \Acro{GNU}     {\acs{GNU} is Not Unix}
  \acro{OO}      {Orientación a Objetos}
  \acro{RPC}     {Remote Procedure Call}
\end{acronym}
}


\section{Instrucciones}

\begin{itemize}
\item \texcmd{acs} (short) muestra el acrónimo propiamente dicho (como: \acs{OO}).
\item \texcmd{acl} (large) muestra el significado del acrónimo (como: \acl{OO}).
\item \texcmd{acf} (full) muestra ambas (como: \acf{OO}).
\end{itemize}

Por lo general debería usar \texcmd{ac}. La primera vez que aparece en cada capítulo, este
comando muestra la versión \texcmd{acf} y en el resto de las ocasiones usa
\texcmd{acs}. Por ejemplo, escribiendo dos veces \texcmd{ac\{RPC\}} se obtiene: \ac{RPC} y
\ac{RPC}.

Toma la precaución de usar \texcmd{acs} en los nombres de las secciones, caption de los
flotantes y en la definición de otros acrónimos en el entorno \texttt{acronym}, pues estos
aparecen primero en las tablas de contenidos, lo que puede provocar problemas.

Si utilizan \texcmd{Acro} en lugar de \texcmd{acro} para definir acrónimos, el acrónimo
aparecerá siempre en su versión corta. Eso es adecuado para acrónimos muy conocidos, como
\ac{GNU}.

\end{document}


% Local Variables:
%   coding: utf-8
%   mode: latex
%   mode: flyspell
%   ispell-local-dictionary: "castellano8"
% End:
