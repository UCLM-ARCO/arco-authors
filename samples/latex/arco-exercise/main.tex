\documentclass{arco-exercise}
\usepackage{arco-listings}

\setlist{noitemsep}

\docTopic{Diseño y Gestión de Redes}
\docCourse{Curso 2012/2012}

\title{Ejercicios de captura y\\análisis básico de tráfico}
\author{David.Villa@uclm.es}
\date{}

\begin{document}
\maketitle

A continuación se propone una lista de pequeñas herramientas de captura que pueden ser
utilizadas para análisis, monitorización y validación de tráfico, y detección de
anomalías. Aunque pueden implementarse en el lenguaje de programación que el alumno desee
se propone el uso de Python. Los programas resultantes deberían funcionar correctamente al
menos en una plataforma GNU/Linux.

\begin{enumerate}
\item Para una red Ethernet, escriba un programa que cuente el número de tramas que
  aparecen en el enlace con la granularidad temporal indicada como parámetro (en
  minutos). Ejemplo de uso:
\begin{console}
$ ./frame-count.py 2
Slot size: 2 min
18:02:  12
18:04:  14
18:06: 150
Capture time: 341.2s
\end{console} %$

\item Para una red Ethernet, escriba un programa que cuente el número de paquetes de cada
  protocolo (níveles red y transporte) durante el tiempo que esté en ejecución. Ejemplo de
  uso:
\begin{console}
$ ./package-type-count.py
Capture time: 123.4s
ARP:  50
IP:  500
UDP:  80
TCP: 420
\end{console} %$

\end{enumerate}

\end{document}

% Local Variables:
%   coding: utf-8
%   fill-column: 90
%   mode: flyspell
%   ispell-local-dictionary: "castellano8"
%   mode: latex
%   TeX-master: "main"
% End:

%  LocalWords:  Fermín
