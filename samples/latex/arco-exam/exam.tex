\documentclass{arco-exam}
\arcoTopic{Redes}
\arcoExamDesc{Final de Febrero (Test eliminatorio)}
\arcoExamDate{09 de febrero de 2006}

% \printanswers

\begin{document}

\arcoExamAdvice{
  Este test eliminatorio consta de \numquestions{} preguntas con un total de \numpoints{}
  puntos. Las respuestas incorrectas no restan.  Sólo una respuesta es correcta a menos
  que se indique algo distinto. Este examen se valora dentro de la nota de teoría con un
  máximo de 15 puntos. Para continuar el examen de teoría deberá responder TODAS las
  preguntas pudiendo fallar un máximo de 4.}

\arcoExamStudentForm

\begin{questions}

\begin{arcoQuestion}{8}
  Enumere y explique la utilidad de los 6 flags que aparecen en la cabecera TCP.

  \arcoSolutionorbox{6}{
    \begin{itemize}
    \item URG: Activa el puntero urgente.
    \item ACK: El segmento contiene información de reconocimiento.
    \item PSH: Los datos deben entregarse inmediatamente
    \item RST: Rechaza un intento de conexión o resetea una conexión activa
    \item SYN: Se utiliza durante el proceso de conexión
    \item FIN: Se utiliza durante el proceso de desconexión
    \end{itemize}
  }

\end{arcoQuestion}

\begin{arcoQuestion}{1}
  El host `A' recibe un paquete del host `B', que se encuentra en una red
    diferente. ¿Cuál es la MAC origen de la trama recibida?

  \choice{La del host `B'}
  \choice{La del host `A'}
  \correctChoice{La de la interfaz \emph{local} del enrutador}
  \choice{La de la interfaz \emph{remota} del enrutador}
\end{arcoQuestion}

\begin{arcoQuestion}{1}
  Marca la afirmación \textbf{falsa} en referencia al entramado:

  \choice{Es el proceso de encapsulación que se realiza en la capa de enlace.}
  \correctChoice{No requiere delimitar el comienzo de cada trama.}
  \choice{Consiste en dividir los paquetes en tramas.}
  \choice{Depende de la tecnología de enlace que se emplee.}
\end{arcoQuestion}

\begin{arcoQuestion}{5}
  Dada la topología de la figura:

  \begin{parts}
    \part[2]
    Indica las direcciones de red de cada nodo

    \arcoSolutionorbox{3}{
      DIRECCIONES
    }

    \part[3]
    Escribe las tablas de rutas

    \arcoSolutionorbox{3}{
      TABLAS
    }
  \end{parts}

\end{arcoQuestion}


\begin{arcoQuestion}{5}
  Dada la topología de la figura:

  \begin{parts}
    \part[2]
    Indica las direcciones de red de cada nodo

    \arcoSolutionorbox{3}{
      DIRECCIONES
    }

    \part[3]
    Escribe las tablas de rutas

    \arcoSolutionorbox{3}{
      TABLAS
    }
  \end{parts}

\end{arcoQuestion}


\begin{arcoQuestion}{5}
  Dada la topología de la figura:

  \begin{parts}
    \part[2]
    Indica las direcciones de red de cada nodo

    \arcoSolutionorbox{3}{
      DIRECCIONES
    }

    \part[3]
    Escribe las tablas de rutas

    \arcoSolutionorbox{3}{
      TABLAS
    }
  \end{parts}

\end{arcoQuestion}


\question
Enunciado
\begin{parts}
\part[2]
Describe the effect on the balloon industry.
\part[3]
Describe the effect on the aircraft industry.
\end{parts}

\begin{arcoQuestion}{5}
  Una figura y su respuesta

\arcoFigureWithAnswer{0.5}
                     {/usr/share/arco/figures/informatica_gray.pdf}
                     {/usr/share/arco/figures/uclm_logo.pdf}
\end{arcoQuestion}


\end{questions}

\end{document}
