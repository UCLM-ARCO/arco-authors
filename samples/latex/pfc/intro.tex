% -*- coding: utf-8 -*-

\chapter{Introducción}

\drop{L}{etra} capital para empezar el capítulo. Este párrafo debería tener al menos cinco
líneas para que la letra capital quede bien. Este párrafo debería tener al menos cinco
líneas para que la letra capital quede bien. Este párrafo debería tener al menos cinco
líneas para que la letra capital quede bien. Este párrafo debería tener al menos cinco
líneas para que la letra capital quede bien.


\section{Una sección}

Casi vacía.


\section{Un cuadro}

Se denominan «tablas» cuando contienen datos con relaciones numéricas. En
general se denominan «cuadros». Si las columnas están bien alineadas, las líneas
verticales estorban más que ayudan (no las pongas). Los cuadros se referencian
de este modo (ver cuadro~\ref{tab:rpc-semantics}).

\begin{table}[htbp]
  \centering
  {\small
  


\begin{tabular}{p{.2\textwidth}p{.2\textwidth}p{.2\textwidth}p{.2\textwidth}}
  \tabheadformat
  \tabhead{Tipo de fallo}   &
  \tabhead{Sin fallos}      &
  \tabhead{Mensaje perdido} &
  \tabhead{Servidor caído}  \\
\hline
\textit{Maybe}         & Ejecuta:   1 & Ejecuta: 0/1        & Ejecuta: 0/1 \\
                       & Resultado: 1 & Resultado: 0        & Resultado: 0 \\
\hline
\textit{Al-least-once} & Ejecuta:   1 & Ejecuta:   $\geq$ 1 & Ejecuta:   $\geq$ 0 \\
                       & Resultado: 1 & Resultado: $\geq$ 1 & Resultado: $\geq$ 0 \\
\hline
\textit{At-most-once}  & Ejecuta:   1 & Ejecuta:   1        & Ejecuta: 0/1 \\
                       & Resultado: 1 & Resultado: 1        & Resultado: 0 \\
\hline
\textit{Exactly-once}  & Ejecuta:   1 & Ejecuta:   1        & Ejecuta:   1 \\
                       & Resultado: 1 & Resultado: 1        & Resultado: 1 \\
\hline
\end{tabular}


% Local variables:
%   coding: utf-8
%   ispell-local-dictionary: "castellano8"
%   TeX-master: "main.tex"
% End:

  }
  \caption[Semánticas de \acs{RPC} en presencia de distintos fallos]
  {Semánticas de \acs{RPC} en presencia de distintos fallos
    (\textsc{Puder}~\cite{puder05:_distr_system_archit})}
  \label{tab:rpc-semantics}
\end{table}


\section{Una figura}

Las figuras se referencian así (ver figura~\ref{fig:informatica}). Recuerda que
no tienen porqué aparecer en el lugar donde se ponen (mira un libro de
verdad). \LaTeX{} las colocará donde mejor queden, lo te empeñes en
contradecirle, él sabe más que tú sobre tipografía.

\begin{figure}[!h]
\begin{center}
\includegraphics[width=0.2\textwidth]{emblema_informatica.pdf}
\caption{Escudo oficial de informática}
\label{fig:informatica}
\end{center}
\end{figure}

Por cierto, los títulos de tablas, figuras y otro elementos flotantes (los
\texttt{caption}) no acaba en punto.


\section{Un listado de código}

Puedes referenciar un listado con~\ref{code:hello}. Éste es un listado flotante,
pero también pueden ser «no flotantes».

\begin{listing}[
  float,
  language = C,
  caption  = {Hola mundo en C},
  label    = code:hello]
#include <stdio.h>
int main(int argc, char *argv[]) {
    puts("Hola mundo\n");
}
\end{listing}


\section{Citas}

Una cita~\cite{design_patterns}. Si estás viendo la versión PDF de este
documento puedes pinchar la cita, es un enlace que lleva a la página de la
bibliografía en la que aparece la entrada correspondiente.



\section{Más texto para que ocupe varias páginas}

\input{lorem-ipsum.txt}



% Local Variables:
% coding: utf-8
% mode: latex
% TeX-master: "main"
% mode: flyspell
% ispell-local-dictionary: "castellano8"
% End:
