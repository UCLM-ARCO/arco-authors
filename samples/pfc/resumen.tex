% -*- coding: utf-8 -*-

\chapter{Resumen}

El resumen debería estar formado por dos o tres párrafos resaltando lo más
destacable del documento. No es una introducción al problema, es decir, debería
incluir los logros más importantes del proyecto. Suele ser más sencillo
escribirlo cuando la memoria está prácticamente terminada. Debería caber en esta
página (es decir, esta cara).

El presente documento es una platilla para la realización de la memoria del
Proyecto Fin de Carrera según el formato y criterios de la Escuela Superior de
Informática de Ciudad Real. El texto en sí mismo es una serie de consejos sobre
tipografía, \LaTeX, redacción y estructura de la memoria que podrían resultar de
ayuda. Por este motivo, se aconseja al lector consultar el código fuente de
este documento.

Si encuentra cualquier error o tiene alguna sugerencia, por favor, utilice
el \emph{issue tracker} del proyecto \texttt{arco-tools} en:

\url{https://arco.esi.uclm.es:3000/projects/arco-tools}


% Local Variables:
%   coding: utf-8
%   mode: latex
%   TeX-master: "main"
%   mode: flyspell
%   ispell-local-dictionary: "castellano8"
% End:
